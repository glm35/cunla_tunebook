
\documentclass[a4paper,12pt]{article}

\usepackage[frenchb]{babel}
\usepackage[T1]{fontenc}
%\usepackage[latin1]{inputenc}
%\usepackage{ucs}
\usepackage[utf8]{inputenc}

\usepackage{syntonly}
%\syntaxonly % A commenter pour générer réellement le document

\usepackage{gchords}


\author{Gwenaël \bsc{Lambrouin}}
\title{Airs irlandais}
%\date{04 janvier 2003}
% Avec le makefile, le TuneBook n'est regénéré que quand il y a une
% modification. Donc on peut bien utiliser la commande \today
\date{\today}


\begin{document}

\maketitle

%\section{Introduction}

Une compilation des airs de musique irlandaise que j'ai appris à la
flûte, et (ou) les airs pour lesquels j'ai cherché un accompagnement à la
guitare.

L'intérêt est de pouvoir réviser régulièrement les morceaux, et de retrouver
rapidement un air ``oublié''.




\pagebreak
\section{The Headless Turkey (Jig)}
\label{003-TheHeadlessTurkey}
\begin{lilypond}
\paper {
  bookTitleMarkup = \markup {
    \fill-line {
      \fromproperty #'header:composer
    }
  }
}
\include "003-TheHeadlessTurkey.ly"
\end{lilypond}

%%INSERT_INDEX

\end{document}
