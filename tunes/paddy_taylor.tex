\chords{
  \chord{12}{1,1,1,x,x,x}{D(h)}
  \chord{2}{x,4,n,1,2,x}{D}
  \chord{2}{x,3,n,1,2,x}{D/C\#}
  \chord{2}{x,1,n,1,2,x}{D/B}
  \chord{3}{3,3,n,n,1,1}{G}
  \chord{2}{x,4,3,1,2,x}{D(2)}
}

\tune

Le premier accord de D, noté D(h), est joué uniquement avec des
harmoniques. A la manière de John Doyle, on peut l'utiliser durant toute
l'exposition du morceau.