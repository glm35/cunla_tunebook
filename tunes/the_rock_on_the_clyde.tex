\tune

Pour finir le morceau, la guitare joue une variante de la mélodie à l'unisson
avec les autres instruments:

\begin[fragment,staffsize=20,line-width=13.5\cm]{lilypond} {
  \key e \minor
  \time 6/8

  \relative e'' {
    e8 d b  a g a | b g e  a fis d | e4.
  }
}
\end{lilypond}

Le 29 août 2018, j'ai vu Bríd Harper en concert avec le joueur de concertina
Tony O'Connell. C'était chez Pascal à La Vigne. J'a demandé à Bríd si c'était
elle qui avait composé ce morceau, et elle m'a dit que non. Elle a appris ce
morceau dans un livre sous le titre "The Rock on the Clyde", et elle ne sait
plus si l'auteur est crédité dans le livre ou si c'est un trad. Dans ce livre,
le morceau  était écrit en Em, mais elle le joue en Dm. D'autres ont repris le
morceau à sa suite en l'appelant "Bríd Harper's", et maintenant souvent on lui
pose la question de savoir si c'est elle qui l'a composée (If I got 10€ each
time someone asks me the question, I would be rich).

