Un chouette morceau appris du guitariste Sean Whelan au stage de musique
irlandaise à Ti Kendalc'h pendant le week-end de Pâques 2005. Le passage du do
aigu est assez délicat pour les instruments mélodiques\ldots

\tune

On joue l'air 3 fois. La première fois et la deuxième fois, je joue un
accompagnement arpégé sur les 4 cordes aigues:

\chords{
  \chord{3}{x,x,3,n,1,1}{G}
  \chord{2}{x,x,n,1,2,x}{D}
  \chord{3}{x,x,3,n,1,1}{G}
  \chord{7}{x,x,n,1,4,x}{D}
  \chord{3}{x,x,n,3,1,1}{G/C}
}

Sur la partie B de la deuxième fois, la basse nous fait à chaque reprise une
petite descente chromatique (sol, fa\#, fa, mi, mib, ré, puis retour au fa),
ce qui crée une tension dans le morceau. A la fin de la deuxième reprise,
arrivée du fa sur la sixième corde.

\chords{
  \chord{3}{x,x,3,n,1,1}{G}
  \chord{3}{x,x,2,n,1,1}{G/F\#}
  \chord{3}{x,x,1,n,1,1}{G/F}
  \chord{t}{x,x,2,n,3,3}{G/E}
  \chord{t}{x,x,1,n,3,3}{G/Eb}
  \chord{3}{1,x,1,n,1,1}{G/F}
}

La tension est à son comble, et résoud sur un bon vieux Sol joué en fanfare
ainsi que tout le reste du morceau:

\chords{
  \GMaj
  \DMaj
}